\chapter{Minimal Subsystems}\label{chapter:minimal}
Since the main topic of this thesis is the entropy of some minimal or proximal dynamical systems, it is not surprising that the following questions are crucial in our investigations:
\begin{center}
{\it Is a given dynamical system minimal?}

\vspace{2.5mm}

{\it Is a given dynamical system proximal?}
\end{center}

The purpose of this chapter is to develop tools allowing us to tackle questions of this type.
%
More specifically, if $(X,G)$ is a dynamical system, (where $X$ is a compact metric space and $G$ is an amenable group) Theorem~\ref{thm:minimal-synd} provides a formula that for a given $x\in X$ allows us to find the number of minimal components of $\closure{Gx}$, as well as a convenient method for finding these minimal subsystems.

This formula is then used to prove that a dynamical system generated by a quasi-Toeplitz configuration is minimal (Lemma \ref{lem:toeplitz-minimal}). 
%
Additionally, as an almost straightforward consequence of Theorem~\ref{thm:minimal-synd}, we obtain that the function assigning to an element $x\in X$, the number of minimal subsystems of $\closure{Gx}$ is lower-semicontiuous with respect to the Weyl pseudometric (Theorem \ref{thm:mxCont}). 
%
Finally, we use Theorem~\ref{thm:minimal-synd} to demonstrate that the dynamical system constructed in the proof of Theorem \ref{thm:KriegerProximal} is proximal, which we believe is especially tricky to derive without using this result.

To state Theorem~\ref{thm:minimal-synd} let us first introduce some useful notation.
%
For an element $x\in X$ define
\[
m(x) \defeq  \mbox{the number of minimal subsystems of $\closure{Gx}$.} 
\]
It is well known that $m(x)$ is always positive (for the proof see Lemma \ref{lem:mx_non-zero}). 
In a discussion at the end of this section we prove that for the case $X=\{0,1\}^\Z$, all the values $1,2,3, \ldots, \infty$ are attainable by $m(x)$. 
%
The formula for computing $m(x)$ involves the notion of a {\bf set of return times}, i.e., for a given $x\in X$ and $A\subseteq X$ we define
\[
N(x,A)\defeq \{g\in G\,:\,gx\in A\}.
\]
Recall that for $Z\subseteq X$ and $\eps>0$ we let $Z^\eps$ to be the $\eps$-neighborhood  of $Z$.
We are now ready to state the main result of this chapter.

\begin{thm}\label{thm:minimal-synd}
If $x\in X$, then
\begin{equation}\label{eq:min_synd}
m(x) = \min\inbrace{|Z|: Z\subseteq \closure{Gx} \text{ and } N(x,Z^{\eps}) \text{ is syndetic for every } \eps>0 }.
\end{equation}
Moreover, the following hold:
\begin{enumerate}[i)]
\item If $N_1,N_2,\ldots,N_{m(x)}$ are all minimal subsystems of $\closure{Gx}$ and $Z=\{z_1,z_2,\ldots,z_{m(x)}\}\subseteq \closure{Gx}$ is such that $z_i\in N_i$ for every $i=1,2,\ldots,m(x)$, then $Z$ realizes the minimum in \eqref{eq:min_synd}.
\item If $Z^\star \subseteq \closure{Gx}$ realizes the minimum in (\ref{eq:min_synd}), then the family of the minimal subsystems of $\closure{Gx}$ is $\{\closure{Gz}:z\in Z^\star\}$.\label{cond:thm_min_synd}
\end{enumerate}
\end{thm}
\noindent 
The first part of the above theorem is a generalisation of \cite[Lemma~1]{DI88}, where the case $G=\Z$ was studied.
%
To make sense of the above theorem one needs to define what ``syndetic'' means.
%
For a formal definition we refer to Section \ref{section:thick_synd} (below) but for an intuitive understanding it is enough to think of syndetic subsets of $G$ as very ``large''.
%
For instance syndetic subsets of $\Z$ are exactly the ones that have uniformly bounded gaps between their elements.
%
Thus, roughly, counting minimal components of $\closure{Gx}$ is equivalent to finding small sets of points $Z \subseteq \closure{Gx}$ such that the orbit of $x$ visits the (arbitrarily) close vicinity of $Z$ with high frequency.




\noindent


\section{Thick and Syndetic Sets}\label{section:thick_synd}
Before we proceed to the proof of Theorem~\ref{thm:minimal-synd} we need to introduce the concept of thick and syndetic sets and prove some of their properties.
\begin{defn}[Thick and syndetic sets] \label{def:synd_thick}
We define the following properties of subsets of $G$
\mbox{}
\begin{enumerate}
\item {\bf (Thick)} A set $T\subseteq G$ is {\bf thick} if for every finite set $F\subseteq G$ there is $g\in G$ with $Fg\subseteq T$.
\item {\bf (Syndetic)} A set $S\subseteq G$ is {\bf syndetic} if there is a finite set $F\subseteq  G$ such that $FS=G$.
\end{enumerate}
\end{defn}

\noindent
More generally one can define left/right thick and left/right syndetic sets, yet in our setting only the above variants are relevant, hence we chose to stick to them and refer to them simply as \emph{thick} and \emph{syndetic}.

\begin{example}[Thick and Syndetic sets in $\Z$]\label{ex:thick_synd_Z}
A set $S\subseteq\Z$ is syndetic if and only if it has uniformly bounded gaps, i.e., there exists $k\in\N$ such that for every $n\in\Z$ one has 
\begin{equation}\label{eq:Zsynd}
\{n,n+1,\ldots,n+k\}\cap S\neq\emptyset.
\end{equation}
Indeed, to see that the latter condition implies that $S$ is syndetic assume that there exists $k\in\Z$ such that (\ref{eq:Zsynd}) holds for every $n\in\Z$. That means that for every $n\in\Z$ there exist $s\in S$ and $j\in\{0,1,\ldots,k\}$ such that $n+j=s$. 
Therefore $\Z=F+S$, where $F=\{-k,-k+1,\ldots,0\}$.

On the other hand, let $F\in\Fin(\Z)$ be such that $F+S=\Z$. Without loss of generality we can assume that $F=\{0,1,\ldots,k\}$ for some $k\in\N$. Fix $n\in\Z$. We show that the condition \eqref{eq:Zsynd} is satisfied. Observe that $n+k= j+s$ for some $s\in S$ and $j\in F$. Then $s=n+k-j$, which implies $s\in\{n,n+1,\ldots,n+k\}$ and the claim follows.

Similarly, one can show that a set $T\subseteq\Z$ is thick if and only if it contains arbitrarily long intervals of consecutive integers i.e., for every $k\in \Z$ there exists $n\in\Z$ such that $\{n,n+1,\ldots,n+k\}\subseteq T$.
\end{example}
\noindent
Example \ref{ex:thick_synd_Z} motivates the following general lemma that captures the duality between these two notions. (Note that this lemma holds rather trivially for $G=\Z$, given the characterisation of thick and syndetic sets provided in the Example \ref{ex:thick_synd_Z}.)

\begin{lem}[Duality between thick and syndetic sets]\label{lem:syndetic-thick}
A set $S\subseteq G$ is syndetic if and only if $S\cap T\neq\emptyset$ for every thick set $T\subseteq G$.
\end{lem}

\begin{proof}
Assume that $S\subseteq G$ is syndetic. Let $F\subseteq G$ be a finite set such that $FS=G$. Fix a thick set $T\subseteq G$. Then there is $g\in G$ such that $F^{-1}g\subseteq T$. Since $FS=G$, there are $f\in F$ and $s\in S$ such that $g=fs$. Thus 
\[
T\supseteq F^{-1}g \ni f^{-1}g = f^{-1}fs=s,
\] 
which means that $s\in S\cap T$.

\noindent
Now assume that $S\subseteq G$ is such that $S\cap T\neq\emptyset$ for every thick $T\subseteq G$. This implies that $G\setminus S$ is not thick. Therefore there exists $F\subseteq G$ finite such that for every $g\in G$ we have $Fg\cap S\neq \emptyset$. We claim that $ F^{-1}S=G$. Take any $h\in G$. Let $f\in F$ be such that $fh\in S$. Then 
\[
h=(f^{-1}f)h=f^{-1}fh\in F^{-1}S. \qedhere
\]
\end{proof}

\begin{rem}
An analogous property holds for thick sets: 
a set $T\subseteq G$ is thick if and only if $S\cap T\neq\emptyset$ for every syndetic set $S\subseteq G$.
\end{rem}
\noindent
The following lemma combined with Lemma \ref{lem:syndetic-thick} is useful in further investigations concerning syndetic sets (see also \cite{DI88}).

\begin{lem}\label{lem:minimalthick}
If $Z\subseteq \closure{Gx}$ is non-empty and invariant\footnote{Recall that a set $A\subseteq X$ is invariant if $GA=\{gx\,:\,g\in G,x\in A\}\subseteq A$.} (not necessarily closed), then for any $\eps>0$ the set $N(x, Z^{\eps})$ is thick.
\end{lem}


\begin{proof}
Let $\eps>0$ and $F\in \Fin(G)$. We show that there exists $g\in G$ such that $Fg\subseteq N(x,Z^{\eps})$. Equivalently, we find $g\in G$ satifiying $fgx\in Z^{\eps}$ for every $f\in F$. Let $\delta>0$ be such that  $\rho(a,b)<\delta$ implies $\rho(fa, fb)<\eps$ for every $f\in F$ and $a,b\in X$. Now take arbitrary $z\in Z$ and choose $g\in G$ such that  $\rho(gx, z)<\delta$. Then $\rho(fgx,fz)<\eps$ for every $f\in F$.  But $fz\in Z$ for every $f\in F$ since $Z$ is $G$-invariant. Therefore $fg x\in Z^{\eps}$ for every $f\in F$.
\end{proof}
\noindent

\section{Counting Minimal Subsystems}
This section is devoted to the proof of Theorem \ref{thm:minimal-synd}. The proof is an adaptation of a proof presented in \cite{DI88}.
\begin{proof}[Proof of Theorem \ref{thm:minimal-synd}]
Let $x\in X$. It is well-known that $m(x)\geq 1$ (see also Lemma \ref{lem:mx_non-zero}).
%
It is most convenient to prove \eqref{eq:min_synd} by showing that the following three conditions are equivalent:
\begin{enumerate}
\item One has $m(x)\leq m$.
\item\label{cond:N-sety} There exist closed $G$-invariant sets $N_1,N_2,\ldots, N_m\subseteq \closure{Gx}$ such that for every $z\in\closure{Gx}$ there exists $i=1,\ldots,m$ satisfying $N_i\subseteq \closure{Gz}$.
\item There exists a set of $m$ points $Z=\{z_1,\ldots, z_m\}\subseteq \closure{Gx}$ such that for every $\eps>0$ the set $N(x, Z^{\eps})$ is syndetic.\label{2}
\end{enumerate}
Note that the equivalence of conditions 1. and 3. proves \eqref{eq:min_synd} by applying the above for $m:=m(x)$ and $m:=m(x)-1$.

To prove that $(1)\Rightarrow (2)$ let $N_1,\ldots, N_{m(x)}$ be all minimal subsystems of $\closure{Gx}$. If $m(x)<m$,  put $N_k:=N_1$ for $k=m(x)+1,m(x)+2,\ldots,m$. Note that for $z\in \closure{Gx}$ the set $\closure{Gz}$ is a subsystem of $\closure{Gx}$ and hence by Lemma \ref{lem:mx_non-zero} it contains a minimal subsystem $N$. Since $N$ is also a minimal subsystem of $\closure{Gx}$, it must be equal to one of $N_i$ for some $i=1,\ldots,m$ (here we use the initial assumption that $n\geq 1$).
Thus $(2)$ holds.

To show that $(2)\Rightarrow (3)$ take $z_i\in N_i$ for $i=1,\ldots,m$, where $N_i$'s are as in $(2)$. Denote $Z=\{z_1,z_2,\ldots,z_m\}$.
Suppose that  $N(x, Z^{\eps})$ is not syndetic. Let $\{F_n\}_{n\in\N}$ be an increasing sequence of finite sets whose union is equal to $G$. Then for every $n\in\N$ one can find $g_n\in G$ such that for every $f\in F_n$ one has $fg_n x\notin Z^{\eps}$. Let $z\in \closure{Gx}$ be a limit point of the sequence $\{g_n x\}_{n\in\N}$. We justify that $\closure{Gz}\cap Z^{\eps}=\emptyset$, which contradicts $(2)$. Fix $h\in G$. Then there exists $N\in\N$ such that $h\in F_n$ for every $n\geq N$. Therefore $hg_nx\notin Z^\eps$ for $n\geq N$ and hence $hz\notin Z^\eps$.

It remains to show that $(3)\Rightarrow(1)$.  
Suppose that $m(x)>m$. Let $Z_1, \ldots Z_{m+1}$ be disjoint minimal subsystems of $\closure{Gx}$.
Choose $\eps>0$ such that for any $i\neq j$ one has $\text{dist}(Z_i,Z_j)>2\eps$. 
%
Notice that for any set $Z\subseteq \closure{Gx}$ consisting of $m$ points $z_1,\ldots, z_m$ there exists $i_0=1,\ldots,m$ such that  $Z^{\eps}$ is disjoint from $Z_{i_0}^{\eps}$. 
%
Therefore $N(x, Z^{\eps})\cap N(x, Z_{i_0}^{\eps})=\emptyset$. 
%
By Lemma~\ref{lem:minimalthick} the set $N(x,Z_{i_0}^{\eps})$ is thick. Hence, by Lemma \ref{lem:syndetic-thick}, $N(x,Z^{\eps})$ is not syndetic contradicting $(3)$. 
%
Therefore we must have $m(x)\leq m$.
This concludes the proof of equivalence of conditions $(1)$, $(2)$ and $(3)$. 

Next, we justify $(i)$.
%
By following the steps $(1)\Rightarrow (2)$ and $(2)\Rightarrow (3)$ one can conclude that if $N_1,N_2,\ldots, N_m$ are minimal subsystems of $\closure{Gx}$ and $Z=\{z_1,z_2,\ldots,z_m\}$ is chosen such that $z_i\in N_i$ for every $i=1,2,\ldots,m$, then $Z$ indeed satisfies the condition $(3)$.

Finally, we prove $(ii)$. Since an orbit of every element in a minimal dynamical system is dense, it remains to show that if we have $Z=\{z_1,\ldots, z_m\}$ as in condition 3. and $Z_1, Z_2,\ldots, Z_m$ are all minimal subsystems of $\closure{Gx}$, then $Z\cap Z_i\neq\emptyset$ for every $i=1,2,\ldots,m$. 
%
Fix $k\in\{1,2,\ldots,m\}$ and $\eps>0$. 
%
Since $N(x,Z_k^{\eps})$ is thick (Lemma \ref{lem:minimalthick}) and $N(x,Z^{\eps})$ is syndetic, we have $N(x,Z_k^{\eps})\cap N(x,Z^{\eps})\neq\emptyset$ (by Lemma \ref{lem:syndetic-thick}).
%
Take $g\in N(x,Z_k^{\eps})\cap N(x,Z^{\eps})$. 
%
Then there exists $l\in\{2,\ldots,m\}$ such that  $\rho(gx, z_l)<\eps$ and  $\dist(gx, Z_k)<\eps$. 
%
This implies $\dist(Z_k,z_l)<2\eps$. But $\eps$ can be arbitrarly small and $Z_k$ is closed, hence $\dist(Z_k,Z)=0$. 
\end{proof}


\subsection*{Example: application of Theorem \ref{thm:minimal-synd}.}
\begin{example}[Application of Theorem \ref{thm:minimal-synd}]
We consider an interesting example that shows how useful Theorem \ref{thm:minimal-synd} might be. Consider the sequence $x_\mbP\in \{0,1\}^\Z$ 
given as
$$
x_\mbP(k) = \begin{cases}
1 & \mbox{if $|k|$ is prime,} \\
0 & \mbox{otherwise.}\end{cases}$$
\noindent We show that the subsystem $\closure{Zx_\mbP}$ is proximal and has only one minimal subsystem: $\{0^\infty\}$.

Note that the subsystem $\closure{Zx_\mbP}$ is rather complicated and tricky to ``understand''. Indeed, even the question of whether $0^\infty 1 0 1 0^\infty$ belongs to $\closure{Zx_\mbP}$ is a hard problem in Number Theory, intimately related to the Twin Prime Conjecture.

Using Theorem~\ref{thm:minimal-synd} our problem of showing that $\closure{Zx_\mbP}$ is proximal boils down to proving that for every $\eps>0$ the set $N(x_\mbP, \{0^\infty\}^\eps)$ is syndetic, which in turn reduces quite simply to proving the following number-theoretic lemma:
\end{example}
\begin{lem}[Large, frequent gaps between primes]
For every number $k\in \N$ there exists $n\in \N$ such that every interval of natural numbers of length $n$ contains $k$ consecutive composite numbers.
\end{lem}
\begin{proof}
Fix $k\in\N$ and let $N:=k+1$. We claim that $n=2N!$ is enough.
To this end observe that each number of the form
\[lN! + j\]
for $l\in \N_{>0}$ and $j=2,3, \ldots, N$ is composite.
\end{proof}

\subsection*{Discussion: values attainable by $\bm{m(x)}$}

We first justify that for any $x\in X$ we have $m(x)>0$.

\begin{lem}\label{lem:mx_non-zero}
For every dynamical system $(X,G)$, there exists a minimal subsystem.
\end{lem}

\begin{proof}
We use Zorn's Lemma. Let $$P=\inbrace{Y\subseteq X: (Y,G) \text{ is a subsystem of } (X,G)}.$$ Then $P$ is partially ordered by inclusion. Clearly, $P\neq\emptyset$ since $X\in P$. Now, take any non-empty descending chain $L\subseteq P$ and denote $Y_0\defeq \bigcap L$. Notice first, that $L$ is a descending family of non-empty compact sets, hence by the Cantor's intersection theorem, we have $Y_0\neq\emptyset$. Moreover, as the intersection of a family of closed sets, $Y_0$ is closed. We justify that it is also $G$-invariant. Let $y\in Y_0$ and $g\in G$. Then $y\in Y$ for every $Y\in L$, which implies $gy\in Y$ for every $Y\in L$. Hence $gy\in Y_0$.  That means that the family $(P,\subseteq)$ satifies the assumptions of Zorn's Lemma. Therefore there exists a minimal element in $P$.
\end{proof}



Now we show that for every $m\in \N\setminus\{0\}$ there exists $x\in \inbrace{0,1}^{\Z}$ such that $m(x)=m$. Denote $[n]=\{1,2,\ldots,n\}$ for $n\in\N$ and let $[a,b]$ be the interval in $\Z$, that is, $[a,b]:=[a,b]\cap\Z$ for $a,b\in\Z$.  

Fix $m\in \N\setminus\{0\}$. 
For $i=1,2,\ldots,m$ let a word $w_i\in \{0,1\}^{[i]}$ be given by 
\[
w_i \defeq 10^{i-1}.
\]
Next define sequences $x,z_1,z_2,\ldots,z_m \in\{0,1\}^{\Z}$ as\footnote{For $x\in\{0,1\}^\Z$,  we separate the $x(i)$ with $i\geq 0$ from those with $i<0$ with a ``decimal point''.}
\begin{align*}
& z_i\defeq \ldots w_iw_i.w_iw_i\ldots  ~~~~\text{ for } i=1,2,\ldots,m, \\
& x\defeq \ldots w_1^2w_2^2\ldots w_m^2 w_1 w_2 \ldots w_m.w_1 w_2 \ldots w_m w_1^2w_2^2\ldots w_m^2 w_1^3 w_2^3 \ldots w_m^3 \ldots
\end{align*}
Clearly, for every $i=1,2,\ldots,m$ one has $\closure{\Z z_i}\subseteq\closure{\Z x}$ and $\closure{\Z z_i}$ is minimal (since $z_i$ is periodic). 
%
Moreover, if $i\neq j$ then $\closure{\Z z_i}\neq \closure{\Z z_j}$, simply because $\closure{\Z z_i}$ consists of $i$ elements and $\closure{\Z z_j}$ consists of $j$ elements. Hence $m(x) \geq m$.

To prove that $\closure{\Z z_i}$'s are the only minimal subshifts, we use Theorem \ref{thm:minimal-synd} with $Z=\{z_1,z_2,\ldots, z_m\}$.
%
Fix $\eps>0$ and $N\in\N$ satisfying $\frac{1}{N}<\eps$. We justify that the set $N(x,Z^{\eps})$ is syndetic. For this, we need to show that there exists $k\in\N$ such that for every $n\in\Z$ one can find $j\in[n,n+k]$ and $l=1,2,\ldots,m$ such that 
\[
x_{[j-N,j+N]}=(z_l)_{[-N,N]}.
\]
In other words, we need to find $k\in\N$ such that in every interval $I\subseteq\Z$ of length $k$, there exists an interval $J\subseteq I$ of length $2N+1$ such that $x_J = (z_l)_{[-N,N]}$ for some $l=1,2,\ldots,m$. Note that for that purpose it is enough to guarantee that in every interval of length $k$ there exists a subword consisting of $2N+1$ consecutive repetitions of some $w_i$ (for any $i\in [m]$). To achieve this, it suffices to pick $k$ to have length at least:
\[
2 \cdot \inmodul{w_1 w_2 \ldots w_m w_1^2w_2^2\ldots w_m^2\ldots w_1^{2N+1}w_2^{2N+1}\ldots w_m^{2N+1}},
\]
to be able to cover the whole ``middle part'' of $x$ where there can be less than $2N+1$ consecutive repetitions of $w_i$'s. Numerically,
\begin{align*}
k&\defeq 2\cdot m\cdot (2N+1)^2
\end{align*}
is clearly sufficient. Therefore $m(x)=m$ holds.

Finally, note that $\{0,1\}^{\Z}$ is transitive and it has infinitely many minimal subshifts, hence there exists $x\in\{0,1\}^{\Z}$ such that $m(x)=\infty$.

\section{Proximal Subsystems}

In this section we use Theorem \ref{thm:minimal-synd} to give a sufficient condition for a dynamical system to be proximal.

\begin{lem}\label{lem:proximal_characterisation}
If a dynamical system $(X,G)$ has a fixed point $z\in X$ that is the unique minimal subsystem, then $(X,G)$ is proximal.
\end{lem}

\begin{proof}
Fix $\eps>0$. Observe first that if $x\in X\setminus\{z\}$, then $\closure{Gx}$ is a subsystem of $X$ such that $\{z\}$ is its unique minimal subsystem. Moreover, by Theorem \ref{thm:minimal-synd} part $(i)$ the set $N(x, \{z\}^{\eps})$ is syndetic and by Lemma \ref{lem:minimalthick} it is thick.

We claim that for any given $x,y\in X$ there exists $g\in G$ such that $\rho(gx,gy)<2\eps$.
%
If either $x$ or $y$ is equal to $z$ then the claim holds trivially, because $z\in \closure{Gx}$ and $z$ is a fixed point.
%
We can thus assume $x,y\in X\setminus \{z\}$ and $x\neq y$. Since every syndetic set has a nonempty intersection with every thick set (see Lemma \ref{lem:syndetic-thick}), there exists an element $g\in G$ such that 
\[
g\in N(x, \{z\}^{\eps})\cap N(y, \{z\}^{\eps}).
\]
Then $\rho(gx,z)\leq\eps$ and $\rho(gy,z)\leq \eps$, which implies $\rho(gx,gy)\leq 2\eps$. \qedhere

\end{proof}

\noindent
Note that in the case when $G=\Z$, the converse of Lemma \ref{lem:proximal_characterisation} is also true (see \cite{AK03}). It is known that in general, if a dynamical system is proximal, then it has a unique minimal subsystem, but it does not need to be a singleton (see \cite{deVries93}). We present below an easy proof of this fact.

\begin{lem}\label{lem:proximal_unique_minimal}
If $(X,G)$ is proximal, then it has a unique minimal subsystem.
\end{lem}

\begin{proof}
Assume for the sake of contradiction that $N_1$ and $N_2$ are distinct minimal subsystems of $X$. Clearly, $N_1$ and $N_2$ are disjoint and closed, hence $\eps:=\dist(N_1,N_2)>0$. Let $x\in N_1$ and $z\in N_2$. Then $\rho(gx,gz)\geq \eps$ for every $g\in G$. This implies $\inf_{g\in G}\rho(gx,gz)\geq \eps$ which is a contradiction with proximality of $(X,G)$.
\end{proof} 

\noindent
Lemmas \ref{lem:proximal_characterisation} and \ref{lem:proximal_unique_minimal} lead to a natural question: Is it always true that the unique minimal subsystem for a proximal action is a singleton of a fixed point?
This inspired a definition of strongly amenable group which was introduced by Glasner in \cite{Glasner76}. A group is called strongly amenable if every its proximal action on a compact space has a fixed point. For example, all countable abelian groups are strongly amenable (see \cite{Glasner76}). It is also known that if a group is not amenable, then it is not strongly amenable (see \cite{Glasner76}). The only known example of a proximal dynamical system $(X,G)$ without a fixed point and such that $G$ is countable is an action of the Thompson's group presented in \cite{HJTV19}. Unfortunately, it is not known whether the Thompson's group is amenable. One can find examples of actions of uncountable groups that are amenable but not strongly amenable in \cite{GW02} and \cite{Glasner83}. It is not clear whether all congruent monotileable groups are strongly amenable.


