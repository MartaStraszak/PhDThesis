Dynamical systems are ubiquitous in all of today's science.
%
Virtually every natural process, whether it is in physics, biology, economics or sociology can be modeled as a dynamical system. 
%
Consequently, there has been a lot of research aimed to understand their mathematical properties and especially their asymptotic behavior. 
%
Although many of the nature-inspired dynamical systems are amenable to a detailed mathematical analysis and thus allow for reliable predictions of future behavior, it became clear quite early on that not all of them behave this way.
%
The first glimpse of that phenomena comes from the work of Poincar\'e.
% 
More recently, in 1961, Lorenz discovered that his model of atmospheric convection is extremely sensitive to initial conditions, thus even deterministic dynamical systems derived from the physical laws of nature may generate chaos. 
%
One of the most important tools in the theory of dynamical systems, and also the main research subject of this thesis, is the notion of entropy, familiar also from physics, information theory and statistics.
%



%nowy akapit:
In this dissertation, we are concerned with discrete dynamical systems that consist of a state-space $X$ and an invertible map $T\colon X\to X$ that specifies how the states evolve in a unit of time. 
%
Another way of thinking about discrete dynamical systems is to identify them with the action of the additive group $\Z$ on $X$.
%
This leads to a generalization, where we replace $\Z$ with an arbitrary countable discrete group $G$.
%
Moreover, it is convenient to additionally assume that $G$ is amenable. This assumption is necessary and natural to generalize basic concepts of ergodic theory such as entropy or Banach density, as well as to prove generalizations of fundamental results such as Krylov–Bogolyubov theorem or the pointwise ergodic theorem.
%


One of the main motivations for this thesis is the following fundamental question about the entropy in the archetypical example of a symbolic dynamical system, the Cantor space $X=\{0,1\}^G$. 
%
Which values between $0$ and $\log 2$ can be realized as the entropy on a particular family of subsystems of $X$? 
%
One attempt of answering this question was made by F. Krieger, who proved via an intricate construction that each value can be realized as the entropy of some minimal subsystem of $X$, provided that $G$ is residually finite.
%
In this thesis, we present  a novel approach to the proof of realizability theorem for minimal subsystems that allows us to extend it from residually finite groups to congruent monotileable groups.
%
In our proof, the crucial realization is that if the space $X$ is equipped with the Weyl metric, then the entropy (thought of as a function mapping $x\in X$ to the topological entropy of the closure of the $G$-orbit of $x$) becomes a continuous function (while the analogous property for the standard metric on $\{0,1\}^G$ obviously fails).
%
In connection to that, we study the interplay between the Weyl metric, invariant measures on $X$ and minimal subsystems of $X$, and obtain generalizations of the results of Downarowicz and Iwanik (who studied the case $G=\Z$). 


As another contribution, we prove a new variant of the realizability theorem that proximal subsystems also achieve all values of entropy.
%
This result we approach from a completely different angle. We develop the theory of subordinate shifts and show that we can control their entropy, manipulating the density of $1$'s in their primary shifts.

